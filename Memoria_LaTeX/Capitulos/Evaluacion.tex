\chapter{Evaluación de la herramienta}
\label{cap:evaluacion}

En este capítulo se escribirá la evaluación sobre nuestra herramienta separando en una evaluación para el módulo de OCR, otra para el módulo de tests y una evaluación de la herramienta de forma genérica.
En la parte de OCR se evaluará la librería utilizada, la mejora que proporciona la parte de preprocesamiento y la limpieza de caractéres basura utilizando distancia Levenshtein.
En la parte de tests se evaluará la precisión de los tests asegurándonos de que no exista casos que de un falso error en el test, la evaluación sera con test de unidad.
De forma genérica se evaluará la herramienta con los resultados obtenidos ejecutando la herramienta midiendo la precisión y la matriz de confusión.

\section{Pruebas y resultados de los OCR}
\label{sec:Evaluación_OCR}
En esta sección se hará una evaluación de los distintos OCRs para elegir la más conveniente para nuestra herramienta.
La evaluación se hará sobre las librerías de Tesseract, EasyOCR y Ocropus, son estas elegidas por su disponibilidad gratuita.

El objetivo de la evaluación es elegir el OCR que mejor se adapte a nuestras necesidades, lo que resalta la importancia de la precisión del OCR.
Para poder medir esa precisión, tendremos en cuenta el CER(\textit{Character Error Rate}). Además el tiempo también es un factor importante que tendremos en cuenta.

Según \cite{CER} el CER(\textit{Character Error Rate}) es una métrica usada comúnmente para evaluar la precisión de un sistema de reconocimiento de caracteres (OCR) u otros sistemas que transcriba texto. El valor se obtiene calculando el número de errores a nivel de carácter dividido por el número total de caracteres en el texto de referencia.
Los errores pueden ser debidos a:
\begin{enumerate}
	\item Sustituciones: Un carácter incorrecto reemplaza a uno correcto.
	\item Inserciones: Aparece un carácter extra que no debería estar.
	\item Eliminaciones: Falta un carácter que debería estar presente.
\end{enumerate}
La fórmula general para calcular el CER es:

$CER = \frac{S+D+I}{N} $ 

Donde:

S es el número de sustituciones.

D es el número de eliminaciones.

I es el número de inserciones.

N es el número total de caracteres en el texto de referencia.

El CER se expresa como un valor entre 0 y 1 (o como porcentaje), donde 0 significa que no hubo errores (transcripción perfecta) y 1 (o 100) indica que todos los caracteres son incorrectos.

La metodología seguida para la evaluación es la siguiente:

\begin{enumerate}
	\item Se hará la evaluación de forma independiente con las 3 librerías de OCR. Tesseract, EasyOCR y Ocropus.
	\item Se ha elegido 13 imágenes nombrados del 1 al 13 siendo la imagen 1(figura \ref{fig:1.png}) el más simple y el 13(figura \ref{fig:13.png}) el más complejo. Para la clasificación de simplificad de las imágenes se ha tenido en cuenta el número de caracteres a reconocer y la complejidad del fondo, entre ellas, el contraste y el número de geometrías.
	Las imágenes se pueden encontrar en el repositorio del trabajo\footnote{(Repositorio github con las imágenes de evaluación)\url{https://github.com/Dewo2000/TFG-2024-2025/tree/main/OCRTests/images}}.
	\begin{figure}[H]
		\centering
		\includegraphics[width = 1\textwidth]{Imagenes/Evaluacion_OCR/1.png}
		\caption{1.png}
		\label{fig:1.png}
	\end{figure}
	\begin{figure}[H]
		\centering
		\includegraphics[width = 1\textwidth]{Imagenes/Evaluacion_OCR/13.png}
		\caption{13.png}
		\label{fig:13.png}
	\end{figure}
	\item Cada imagen será procesadas por el OCR utilizando el modelo por defecto de cada una de ellas.  
	\item Se utiliza la libreria Jiwer\footnote{(Jiwer Usage)\url {https://jitsi.github.io/jiwer/usage/}} en python para los cálculos de CER.
	\item Se ejecutará 6 veces el proceso de reconocimiento de cada OCR para sacar el tiempo medio de cada una.
\end{enumerate}


En los resultados de los ocr se encontrará un valor llamado CER medio, este valor se obtiene calculando el número total de errores a nivel de carácter encontrado en toda la batería de prueba divido entre el número total de caracteres de toda la batería de prueba.

Los resultados de los OCRs son las siguientes:
\begin{table}[H]
	\centering
	\caption{Resultado de CER de los OCR (Redondeado a 3 decimales)}
	\begin{tabular}{llll}
		\textbf{Imagen} & \textbf{Tesseract} & \textbf{EasyOCR} &\textbf{Ocropus} \\
		1  & 0.000 & 0.000 & 0.000 \\
		2  & 0.063 & 0.281 & 0.406 \\
		3  & 0.538 & 0.212 & 0.538 \\
		4  & 0.043 & 0.109 & 0.978 \\
		5  & 0.289 & 0.086 & 0.719 \\
		6  & 0.025 & 0.025 & 0.700 \\
		7  & 0.750 & 0.000 & 1.417 \\
		8  & 0.935 & 0.652 & 0.870 \\
		9  & 0.721 & 0.256 & 0.744 \\
		10 & 1.000 & 0.857 & 8.286 \\
		11 & 0.050 & 0.150 & 2.550 \\
		12 & 0.352 & 0.055 & 0.834 \\
		13 & 0.230 & 0.337 & 0.881 \\
		\textbf{CER medio} & \textbf{0.384}& \textbf{0.232} & \textbf{1.456}\\
	\end{tabular}
	\label{table:TesseractResult}
\end{table}
\begin{table}[H]
	\centering
	\caption{Resultado de OCR en tiempo(ms)}
	\begin{tabular}{llll}
		\textbf{Iteración} & \textbf{Tesseract}& \textbf{EasyOCR}& \textbf{Ocropus} \\
		1  & 5539   & 204625 & 86213  \\
		2  & 5580   & 223809 & 90172  \\
		3  & 5539   & 216701 & 88290  \\
		4  & 5720   & 238900 & 83278  \\
		5  & 5602   & 216791 & 87810  \\
		6  & 5571   & 204502 & 88789  \\
		\textbf{Tiempo medio} & \textbf{5591}&\textbf{217554}&\textbf{87425} \\
	\end{tabular}
	\label{table:TesseractResultTime}
\end{table}


Como podemos ver en los resultados, todos los OCRs han podido reconocer de forma exacta la imagen más simple y que todos han caido en la imagén 10. En cuanto las otras imágenes podemos observar que unas lo reconoce mejor Tesseract y otras EasyOCR por lo que ambos producen buenos resultados, en cuanto Ocropus, sus resultados ya no son tan buenos en comparación con las otras dos por lo que queda descartado. Aunque EasyOCR obtiene un mejor CER medio en los resultados (0.23) que Tesseract (0.38), Tesseract es muchísimo más rápido que EasyOCR teniendo un 211936 milisegundo de adelanto que equivale a 3 minutos y 50 segundos aproximadamente. Por tanto decidimos seguir en adelante con Tesseract.



\section{Mejoras en el reconocimiento de texto en imágenes}
\label{sec:Mejoras en el reconocimiento}
Como podemos notar en los resultados del apartado anterior, la imagen 10 ha sido un reto para el CER de los OCRs, esto es debido a que el OCR reconoce como carácter geometrías del fondo y produce basura.Para solucionar esto, se plantea aplicar preprocesamiento a las imágenes y la limpieza del resultado usando distancia Levenshtein para mejorar la precisión del texto obtenido por el OCR disminuyendo el CER.

Para obtener más información sobre aquellos factores o características de las imágenes que puedan afectar al rendimiento y efectividad de la herramienta de OCR, se ha clasificado las imágenes en 5 categorías diferentes. La clasificación se ha hecho a ojo humano de forma subjetiva.
\begin{enumerate}
	\item Fondos simples(F.Simple)(Figura \ref{fig:Fsimple}). Imágenes donde la cantidad de geometrías del fondo son pocas o fáciles de reconocer(no se confunde con los caractéres).
	\begin{figure}[H]
		\centering
		\includegraphics[width = 0.5\textwidth]{Imagenes/OCR/Simple.png}
		\caption{Ejemplo imagen fondos simples }
		\label{fig:Fsimple}
	\end{figure}
	
	\item Fondos complejos(F.Complejo)(Figura \ref{fig:Fcomplejo}). Imágenes donde la cantidad de geometrías del fondo son muchas o difíciles de reconocer(fácil de confundirse con los caractéres).
	\begin{figure}[H]
		\centering
		\includegraphics[width = 0.5\textwidth]{Imagenes/OCR/Complejo.png}
		\caption{Ejemplo imagen fondos complejos }
		\label{fig:Fcomplejo}
	\end{figure}
	\item PixelArt(PixelArt)(Figura \ref{fig:Pixelart}). Imágenes donde el fondo y las letras tiene un estilo pixelart. Como las imágenes son píxeladas(cuadrado de color que forman formas y texto), supone un reto para el OCR.
	\begin{figure}[H]
		\centering
		\includegraphics[width = 0.5\textwidth]{Imagenes/OCR/Pixel.png}
		\caption{Ejemplo imagen PixelArt }
		\label{fig:Pixelart}
	\end{figure}
	\item Texto en bocadillos(TxtBoc)(Figura \ref{fig:TxtBoc}). Imágenes donde el texto a reconocer se situa en un bocadillo y que el color del bocadillo tiene un alto contraste con el fondo.
	\begin{figure}[H]
		\centering
		\includegraphics[width = 0.5\textwidth]{Imagenes/OCR/Boc.png}
		\caption{Ejemplo imagen de texto en bocadillo }
		\label{fig:TxTBoc}
	\end{figure}
	\item Texto en bocadillos con poca diferenciación con el fondo(TxtBoc2)(Figura \ref{fig:TxtBoc2}).Imágenes donde el texto a reconocer se situa en un bocadillo y que el color del bocadillo tiene un contraste medio o bajo con el fondo.
	\begin{figure}[H]
		\centering
		\includegraphics[width = 0.5\textwidth]{Imagenes/OCR/Boc2.png}
		\caption{Ejemplo imagen de texto en bocadillos con poca diferenciación con el fondo }
		\label{fig:TxtBoc2}
	\end{figure}
\end{enumerate} 


El preprocesamiento de imágenes consiste en aplicar técnicas que modifican las imágenes para que sea más fácil de reconocer por una OCR el texto existente.
Para saber que técnicas mejoran el reconocimiento, se ha ido probando con cada una de ellas en el manual de \cite{OpenCVProcessing}, obteniendo los resultados e identificando aquellas técnicas que ayuda a mejorar el CER medio.
Las técnicas probadas son las siguientes:

\begin{enumerate}
	\item Escala de grises(Figura \ref{fig:EscalaGrises}):
	Convierte una imagen a un formato de un solo canal, representando solo la intensidad de luz. Es útil para simplificar y reducir la cantidad de datos cuando el color no es relevante.
	\begin{figure}[H]
		\centering
		\includegraphics[width = 0.5\textwidth]{Imagenes/Preprocesado/1.png}
		\caption{Imagen a escala de grises}
		\label{fig:EscalaGrises}
	\end{figure}
	\item Aumentar contraste(Figura \ref{fig:Contraste}): 
	Mejora la diferencia entre las áreas claras y oscuras en una imagen, lo que facilita la detección de detalles.
	\begin{figure}[H]
		\centering
		\includegraphics[width = 0.5\textwidth]{Imagenes/Preprocesado/2.png}
		\caption{Imagen con aumento de contraste}
		\label{fig:Contraste}
	\end{figure}
	\item Ecualización del histograma(Figura \ref{fig:Histograma}):
	Ajusta el contraste de la imagen extendiendo la distribución de los niveles de gris, mejorando el rango dinámico y resaltando los detalles.
	\begin{figure}[H]
		\centering
		\includegraphics[width = 0.5\textwidth]{Imagenes/Preprocesado/3.png}
		\caption{Imagen con ecualización de histograma}
		\label{fig:Histograma}
	\end{figure}
	
	\item Corrección del gamma(Figura \ref{fig:Gamma}):
	Ajusta los valores de intensidad en la imagen para compensar la percepción humana y los errores del sensor, lo que puede hacer que ciertas áreas sean más visibles.
	\begin{figure}[H]
		\centering
		\includegraphics[width = 0.5\textwidth]{Imagenes/Preprocesado/4.png}
		\caption{Imagen con corrección del gamma}
		\label{fig:Gamma}
	\end{figure}
	
	\item Filtro de nitidez(Figura \ref{fig:F.Nitidez}): 
	Realza los bordes de una imagen para destacar detalles, útil en aplicaciones donde se requiere mayor definición.
	\begin{figure}[H]
		\centering
		\includegraphics[width = 0.5\textwidth]{Imagenes/Preprocesado/5.png}
		\caption{Imagen con filtro de nitidez}
		\label{fig:F.Nitidez}
	\end{figure}
	
	\item Adaptive Thresholding(Figura \ref{fig:Thresholding}):
	Segmenta una imagen dividiéndola en áreas claras y oscuras, aplicando un umbral que se ajusta de forma adaptativa a las variaciones locales de luz.
	\begin{figure}[H]
		\centering
		\includegraphics[width = 0.5\textwidth]{Imagenes/Preprocesado/6.png}
		\caption{Imagen aplicando adaptive thresholding}
		\label{fig:Thresholding}
	\end{figure}
	
	\item Simple Thresholding(Figura \ref{fig:S.Threshold}): 
	Asigna un valor binario a cada píxel dependiendo de si está por encima o por debajo de un umbral específico, útil para crear máscaras y segmentación sencilla.
	\begin{figure}[H]
		\centering
		\includegraphics[width = 0.5\textwidth]{Imagenes/Preprocesado/7.png}
		\caption{Imagen aplicando simple thresholding}
		\label{fig:S.Threshold}
	\end{figure}
	
	\item Image Blurring (Desenfoque de Imagen)(Figura \ref{fig:Blurring}): 
	Reduce el ruido y los detalles mediante técnicas como filtros Gaussianos o de promediado, comúnmente utilizado para suavizar imágenes antes de un análisis.
	\begin{figure}[H]
		\centering
		\includegraphics[width = 0.5\textwidth]{Imagenes/Preprocesado/8.png}
		\caption{Imagen aplicando desenfoque de imagen}
		\label{fig:Blurring}
	\end{figure}
	
	\item Redimensionar la imagen: 
	Cambia las dimensiones de una imagen, lo que puede ser útil para normalizar entradas a una red neuronal o ajustar el tamaño de una imagen para procesamiento.
	
	\item Dilatar y erosionar(Figura \ref{fig:Dilate_Erode}): 
	Técnicas de morfología matemática que expanden o reducen las regiones blancas (o los objetos) en una imagen binaria, útiles para limpieza de ruido o cierre de contornos.
	\begin{figure}[H]
		\centering
		\includegraphics[width = 0.5\textwidth]{Imagenes/Preprocesado/10.png}
		\caption{Imagen dilatado y erosionado}
		\label{fig:Dilate_Erode}
	\end{figure}
	
	\item Denoising (Reducción de ruido)(Figura \ref{fig:Denoising}): 
	Elimina o reduce el ruido en una imagen para mejorar la calidad visual y el rendimiento de tareas de reconocimiento.
	\begin{figure}[H]
		\centering
		\includegraphics[width = 0.5\textwidth]{Imagenes/Preprocesado/11.png}
		\caption{Imagen aplicando reducción de ruido}
		\label{fig:Denoising}
	\end{figure}
	
\end{enumerate}
A continuación se ha hecho el experimento de aplicar un solo tipo de preprocesamiento a todas las imágenes de cada categoría, ejecutando el OCR sobre las imágenes preprocesadas y obteniendo el CER de cada imágen y el CER medio de cada categoría. Este proceso se ha aplicado para cada uno de las posibles técnicas de preprocesamiento obteniendo el resultado en la tabla \ref{table:preproCERtable}. Cada columna representa la categoría de la imagen y cada fila representa el preprocesamiento aplicado a las imágenes. En este caso no se hace distinción de OCR ya que preprocesamiento se aplica directamente en las imágenes antes de ser pasados a la OCR por lo que no importa la librería de OCR que se este usando.
Se marca en rojo aquel tipo de preprocesamiento que da peor resultado en la categoría y en verde, aquel que da mejor resultado.
\begin{table}[H]
	\begin{tabular}{llllll}
		Tipo                                                                 & F.Complejo                      & F.Simple                      & PixelArt                      & TxTBoc                       & TxtBoc2                       \\
		Nada                                                               & 9.39                          & 1.05                         & 2.51                          & 2.02                         & 5.61                          \\
		Grises                                                               & 3.72                          & 0.59                         & 2.69                          & 1.17                         & 2.16                          \\
		Contraste                                                            & 5.06                          & 0.81                         & 2.39                          & 1.13                         & 10.07                         \\
		\begin{tabular}[c]{@{}l@{}}Ecualización\\ de histograma\end{tabular} & 5.59                          & 3.60                         & 6.56                          & 2.41                         & 8.67                          \\
		Gamma                                                                & 3.72                          & 0.59                         & 2.69                          & 1.17                         & 2.16                          \\
		Filtro de nitidez                                                    & 11.06                         & 2.17                         & 8.84                          & 3.29                         & 12.05                         \\
		Thresholding C                                                       & \cellcolor[HTML]{FF0000}16.80 & \cellcolor[HTML]{FF0000}9.44 & 10.36                         & 4.08                         & \cellcolor[HTML]{FF0000}23.44 \\
		\begin{tabular}[c]{@{}l@{}}Thresholding\\ Gaussian\end{tabular}      & 12.85                         & 9.35                         & \cellcolor[HTML]{FF0000}16.16 & \cellcolor[HTML]{FF0000}4.71 & 22.83                         \\
		Thresold Binary                                                      & 4.05                          & \cellcolor[HTML]{00FF00}0.49 & 2.14                          & 1.79                         & 2.04                          \\
		\begin{tabular}[c]{@{}l@{}}Redimension\\ x1.5\end{tabular}           & 4.77                          & 0.80                         & 2.86                          & 1.27                         & 3.07                          \\
		Gaussian Blur                                                        & 3.02                          & 0.68                         & 2.34                          & 1.14                         & 1.86                          \\
		Median Blur                                                          & 2.75                          & 0.51                         & 2.75                          & 1.25                         & 1.37                          \\
		2 Blur                                                               & 2.26                          & 0.59                         & 2.15                          & 1.22                         & \cellcolor[HTML]{00FF00}1.09  \\
		Dilatación y erosión                                                 & 2.19                          & 0.51                         & 2.48                          & \cellcolor[HTML]{00FF00}0.69 & 1.83                          \\
		Dilatación                                                           & \cellcolor[HTML]{00FF00}1.80  & 0.62                         & \cellcolor[HTML]{00FF00}1.23  & 0.80                         & 5.53                          \\
		Erosión                                                              & 2.02                          & 1.00                         & 2.57                          & 0.82                         & 1.21                          \\
		Denoising                                                            & 3.23                          & 0.57                         & 2.64                          & 1.07                         & 1.70                         
	\end{tabular}
	\caption{Tabla con los resultados de CER medio de cada categoría de imágenes después de aplicar un tipo de preprocesamiento.}
	\label{table:preproCERtable}
\end{table}

Como podemos observar, en la mayoría de los preprocesamientos, los resultados mejoran en comparación con la de sin aplicar nada. Sin embargo, hay otros que empeoran, esto es debido a que el preprocesamiento ha marcado más las líneas y las geometrías por lo que el OCR reconozca más caracteres ``basura'' por lo que el CER supera a 1(reconoce más caracteres de los que hay en el ground-truth). Esto no significa que el OCR vaya ir a peor, puede que reconozca mejor el texto esperado pero añadiendo más basura que antes, algo que intentaremos resolver en el siguiente apartado. 

Obteniendo esta tabla y viendo los resultados de imágenes se ha ido probando distintas combinaciones de preprocesados de imágenes:
\begin{enumerate}
	\item Experimento 1: 
	\begin{itemize}
		\item Grises \item Escalado\item Adaptive Threshold \item Denoising \item Blurring \item Dilate\_Erode    
	\end{itemize}
		\item Experimento 2: 
	\begin{itemize}
		\item Grises\item Escalado\item Denoising Blurring  \item Dilate\_Erode\item Ec. Histograma \item Gamma \item F.Nitidez           
	\end{itemize}
		\item Experimento 3: 
	\begin{itemize}
		\item Grises \item Escalado\item Simple Threshold \item Denoising \item Blurring \item F.Nitidez \item Dilate\_Erode    
	\end{itemize}
		\item Experimento 4: 
	\begin{itemize}
		\item Grises \item Escalado\item Simple Threshold \item Denoising \item Blurring \item Dilate\_Erode    
	\end{itemize}
\end{enumerate}
Obteniendo estos resultados:
\begin{table}[H]
	\begin{tabular}{llllll}
		Experimento & Complejo & Simple & PixelArt & TxTBoc & TxtBoc2                      \\
		 1 & 18.79     & 3.96   & 13.93     & 5.26   & 22.01 \\
		 2 & 7.23     & 4.58   & 14.96     & 4.94   & 11.94 \\
		 3 & 3.68     & 0.34   & 2.30     & 1.16   & 1.62 \\
		 4 & 2.57     & 0.29   & 2.01     & 1.01   & 1.49
	\end{tabular}
	\caption{Tabla con los resultados CER de cada categoría en cada experimento.}
	\label{table:Prepro}
\end{table}
Donde podemos ver en la tabla \ref{table:Prepro} que el experimento más destacado es el experimento 4, por lo que en adelante seguiremos el preprocesamiento con las técnicas del experimento 4.
\subsection{Eliminación de caracteres basura}
\label{subsec:Eliminación de caracter basura}
Obteniendo el resultado de CER medio de los preprocesamientos(tabla \ref{table:preproCERtable}), podemos ver que se produce muchos números que superan al 1, esto significa que el OCR ha reconocido más caracteres de lo que hay en el ground-truth, todos esos caracteres que sobran son caracteres basura y el texto que necesitaremos estará en algunas de esas líneas. En esta sección intentaremos eliminar esos caracteres basura dejando solo lo necesario.
\begin{figure}[H]
	\centering
	\includegraphics[width = 1\textwidth]{Imagenes/Sample_Trash_Char.png}
	\caption{Imagen ejemplo.}
	\label{fig:Trash_Char}
\end{figure}

El ground-truth de la figura \ref{fig:Trash_Char} es la siguiente:
\begin{verbatim}
	Jonny
	If you two hadn't met at the reunion,
	you wouldn't be going out right now.
\end{verbatim}
Y una posible salida de la OCR con la figura \ref{fig:Trash_Char} puede ser la siguiente:
\begin{verbatim}
T A W
Wml"lll"|||"lll“1]]“11‘“"1“l‘l]1"]]]"]]1"1V‘“"‘”V"”””””“WWHH\HH\HH\HH}}}}111111111”‘””}"“H‘H”H‘mwwmm‘ [y " |
G
wiui:lfifl}m;,;r}l’“
Nul1l1lmu11111um111llumI11llumIIIIIIMMI1IllMmlllIIIUU11111qu1I1IIIHH111111HMIIIIIIMMIII1llMMIIIIIIIIII11IIIIIIIIIIIIIIIIIIIIIII1HmIIIIIMIMIIII1“MIIIIIIIIIIII11||m"||11IIMMIIlll“i“1IIIIIMMllllilflU‘Wl“i“ii111HM11111IIUIiiiii“i“ii11IIIIIIIIIIIIIIIIIIIIIIII:“‘ I M = IWHIIIWWII"IIIIIW
Hunum;.x:flwww'M;‘dm‘w‘w‘w‘w‘uHfl“fl“flflmw e 4 ‘ w}\‘s.*"‘r Ll e e p
: L T Q1 A it e
- ' diiry el R ;4
o wm W el 5
i " :_.“;»"*‘ i i m?“ Q:‘]m}illl}jjw ! “hu‘11111111{}}““;11H5:1”\\\\!\\1\11\\\””‘HHHW “hm*' ]
T i 1‘11“1””“””““” HHHHHHH ‘HHU”“HH\HHHHH\HMH\H ‘ (TR
I e > h ¢ 4 ' i
= 4 o
T
- i e
§ ...
E ‘E | \ i
P S y:
,,|,- |
IR ill ‘
— ' A 000 Ir yo
\end{verbatim}

Donde de forma subjetiva no encontramos ningún trozo del texto que se corresponda o sea parecido al ground-truth.

Después de aplicar el preprocesamiento elegido en el apartado anterior obtenemos el siguiente resultado:

\begin{verbatim}
T A W
Wml"lll"|||"lll“1]]“11‘“"1“l‘l]1"]]]"]]1"1V‘“"‘”V"”””””“WWHH\HH\HH\HH}}}}111111111”‘””}"“H‘H”H‘mwwmm‘ [y " G
dJonny If you two hadn't met at the reunion you wouldn't be going out right now
,,|,- |
IR ill ‘
— ' A 000 Ir yo
\centering
\end{verbatim}

En esta salida ya podemos reconocer que hay una línea que se corresponde con la salida que esperamos.
La frase que realmente importa es solamente una o dos líneas de la salida
\begin{verbatim}
Jonny If you two hadn't met at the reunion
you wouldn't be going out right now
\end{verbatim}
Esto es un problema importante que debemos solucionar ya que es imposible saber si un test es correcto o no con esta entrada.

En esta sección se propone una solución a este problema de caracteres ``basura'' usando el algoritmo de distancia \emph{levenshtein}.

La distancia de Levenshtein(también conocida como distancia de edición) según un articulo de \cite{LevDistance}, es una métrica utilizada para medir el grado de diferencia entre dos cadenas de texto. Específicamente, se define como el número mínimo de operaciones necesarias para transformar una cadena en otra, utilizando tres tipos de operaciones básicas:
\begin{itemize}
\item Inserciones: Agregar un carácter.
\item Eliminaciones: Eliminar un carácter.
\item Sustituciones: Reemplazar un carácter por otro.
\end{itemize}

Suponiendo que tenemos el texto esperado de la imagen,  utilizando esta métrica, podemos obtener la distancia levenshtein entre una línea del texto esperado y una línea del texto reconocido por la OCR. Con la distancia obtenida y aplicando un cierto umbral, podemos identificar aquellas líneas que más se asimila al texto esperado, obteniendo así las líneas deseadas y descartando aquellas que no cumpla un cierto umbral.

Uno de los resultados obtenidos aplicando la distancia levenshtein es la siguiente si aplicamos un umbral de similitud de 0.8 (tiene que ser 80\% de parecido):
\begin{itemize}
\item Texto real de OCR:
\begin{verbatim}
"
|
\ J
—
Agumon /
Still, a “smahrt fown” is cool! It really can
do anything! <
\end{verbatim}
\item Texto esperado:
\begin{verbatim}
Agumon
Still,a "smahrt fown" is cool! It really can
do anything!
\end{verbatim}
La fórmula general para calcular la similitud es:

$Simulitud = 1-\frac{d}{max(s1,s2)} $ 

Donde:

\textit{d} es la distancia levenshtein.

\textit{max(s1,s2)}	 es el máximo entre la longitud de la cadena \textit{s1} y cadena \textit{s2}

Empezamos comparando la primera línea de OCR con la primera línea del esperado y así todo el rato hasta que comparemos todas.
\begin{enumerate}
	\item Se compara ``Agumon'' con la primera línea  obteniendo una distacia levenshtein de 6(1 sustitución para sustituir el único carácter a ``A'' y 5 inserciones ``gumon''  ), aplicando la fórmula obtenemos una similitud de 0.
	\item Se compara con las siguientes 3 líneas obteniendo resultado de todos distancia 6 y similitud 0.
	\item Se compara con la siguiente obteniendo una distancia de 1, aplicando la formula obtenemos una similitud de 0.86, por lo que se marca como encontrado y no se vuelve a comparar las líneas encontradas.
	\item Se coge la siguiente línea a comparar y se repite el proceso hasta acabar con todas.
\end{enumerate}
Después de todo ese proceso obtenemos la cadena limpiada utilizando distancia levenshtein.
\item Texto aplicando distancia levenshtein:
\begin{verbatim}
Agumon /
Still "smahrt fown" is cool! It really can 
do anything! <
\end{verbatim}
\end{itemize}  
Aplicando levenshtein a los resultados de todas las imágenes de cada categoría mencionadas en el apartado anterior después de aplicar preprocesamiento obtenemos estos nuevos resultados del CER medio de cada categoría:

\begin{table}[H]
\begin{tabular}{llllll}
Tipo        & Complejo & Simple & PixelArt & TxTBoc & TxtBoc2                      \\
OCR         & 2.57     & 0.29   & 2.01     & 1.01   & \cellcolor[HTML]{FFFFFF}1.49 \\
Levenshtein & 0.63     & 0.16   & 0.63     & 0.55   & 0.42                        
\end{tabular}
\end{table}
El resultado de OCR es el resultado que obtenemos con solamente aplicando el preprocesamiento del apartado anterior. Sobre esa salida, se aplica la distancia levenshtein para la eliminación de basura obteniendo los nuevos resultados. Podemos ver en los números que mejoran casi un 50\% en todas la categorías por lo que es una técnica viable para nuestra herramienta.
\section{Evaluación de los tests}
En esta sección se hará una evaluación de los tests implementados.
El objetivo de esta evaluación es asegurar de que los tests implementados sean correctos sin producir ningún falso positivo y falso negativo.
La metodología de la evaluación es utilizando test de unidad donde existirá una suite de test para casos positivos de cada test y casos negativos de cada test. El test de unidad se puede encontrar en \texttt{unittest.cpp}.

Ejecutando el test de unidad obtenemos que se ha pasado todos los tests, tanto positivos como negativos por lo que podemos llegar a la conclusión de que la implementación de los tests son correctas.

\section{Evaluación de la herramienta}
En esta sección se hará una evaluación de la herramienta de forma genérica incluyendo los dos modulos de OCR y tests.
El objetivo de esta evaluación es asegurar de que nuestra herramienta es eficiente y preciso como ayuda de la parte de LQA.
La metodología de la evaluación es la siguiente:
\begin{enumerate}
	\item 30 imágenes en total. Donde hay 10 imágenes con errores de localización, y los 25 restantes sin error.
	\item Se ejecutara la herramienta propocionando información de estas imagenes y configuración.
	\item Se obtendrá los resultados de las imágenes.
	\item Generar una matriz de confusión con los resultados obtenidos y esperados.
	\item Obtener información de la matriz de confusión como precisión, falsos positivos y falsos negativos.
\end{enumerate}
El resultado de la evaluación es la siguiente:
TABLA
