\chapter{Estado de la Cuestión}
\label{cap:estadoDeLaCuestion}

%%En el estado de la cuestión es donde aparecen gran parte de las referencias bibliográficas del trabajo. Una de las formas más cómodas de gestionar la bibliografía en {\LaTeX} es utilizando \textbf{bibtex}. Las entradas bibliográficas deben estar en un fichero con extensión \textit{.bib} (con esta plantilla se proporciona el fichero biblio.bib, donde están las entradas referenciadas más abajo). Cada entrada bibliográfica tiene una clave que permite referenciarla desde cualquier parte del texto con los siguiente comandos:%

%\begin{itemize}
%\item Referencia bibliografica con cite: \cite{ldesc2e}
%\item Referencia bibliográfica con citep: \citep{notsoshort}
%\item Referencia bibliográfica con citet: \citet{latexAPrimer}
%\end{itemize}

%Es posible citar más de una fuente, como por ejemplo \citep{latexCompanion,LaTeXLamport,texKnuth}

%Después, \LaTeX se ocupa de rellenar la sección de bibliografía con las entradas \textbf{que hayan sido citadas} (es decir, no con todas las entradas que hay en el .bib, sino sólo con aquellas que se hayan citado en alguna parte del texto).

%Bibtex es un programa separado de latex, pdflatex o cualquier otra cosa que se use para compilar los .tex, de manera que para que se rellene correctamente la sección de bibliografía es necesario compilar primero el trabajo (a veces es necesario compilarlo dos veces), compilar después con bibtex, y volver a compilar otra vez el trabajo (de nuevo, puede ser necesario compilarlo dos veces). 

\section{Internacionalización}
La internacionalización es el proceso de preparar un producto para que pueda admitir diferentes idiomas y convenciones culturales sin necesidad de volver a rediseñarse.(Localisation Industry Standards Association)(LISA).

Separación del texto traducible de código fuente: Esto ayuda a que el traductor se centre únicamente en el trabajo de traducción y no tenga que acceder al código fuente.

Uso de fuentes adecuadas para distintos idiomas: Las fuentes deben mostrar todos los posibles caracteres del idioma seleccionado, el fallo de esto puede conllevar a que en el videojuego no se vea el texto completo.

Uso de la codificación adecuada para distintos idiomas: Es necesario que la codificación pueda interpretar un carácter o símbolo introducido por teclado. Se usa el estándar de Unicode.

Diseño de interfaces adaptables al texto que se debe mostrar: La misma cadena de texto en distintos idiomas traducidos pueden necesitar distintos espacios para mostrarlo en pantalla, esto puede conllevar a un problema a la hora de diseñar la interfaz y el espacio que tiene en el videojuego para mostrar una cadena.
Se plantean las siguientes posibilidades de solucionar el problema

\begin{itemize}
\item Uso de fuente de ancho variable siempre que sea posible: La fuente elegida afecta mucho a la hora de mostrar más o menos caracteres. Existen las fuentes de ancho fija (todos los caracteres ocupan exactamente el mismo número de píxeles), fuentes de ancho variable (cada carácter ocupa un determinado número de píxeles). Poner ejemplos.
\item Uso de bocadillos, cajas o ventanas de texto adaptables al contenido: El tamaño de las ventanas se adaptará al tamaño de texto que hay dentro.
\item Uso de menús y botones con gran espacio o adaptables al contenido: En muchas ocasiones surge el problema de que el texto traducido no quepa en el espacio del botón o del menú, para ello se recomienda que se hagan con espacio suficiente para abordar cualquier posible tamaño de la traducción o que sean adaptables al contenido.
\end{itemize}

Uso de etiquetas especiales para marcar género, sexo o número: Un problema a la hora de traducir reside en la concordancia de género, sexo y número en distintos idiomas, un ejemplo es \textit{``You are so nice''} se puede traducir como ``Eres muy simpático'' o ``Eres muy simpática''. Por lo tanto es necesario la existencia de un sistema que cambie la palabra según si se trata de un personaje masculino o femenino.

Facilitación de un mínimo de información contextual: En distintos idiomas puede resultar que la traducción sea diferente dependiendo del contexto, esto el traductor no lo sabe, por lo tanto es necesario un mínimo de información contextual para agilizar el proceso de traducción.

Comprobación cultural de imágenes e iconos: Algunos gestos, imágenes o iconos pueden resultar inapropiados para algunas culturas o religiones hay que tener especial cuidado con la zona de localización.

\section{Localización}

La \textbf{Localización} (o L10N), es el proceso por el cuál un producto audiovisual se adapta a un idioma y/o cultura diferente del idioma o cultura en el que el producto fue creado. Esto incluye, pero no se limita, a la traducción de los textos y audios. Aunque puede parecer que esto es lo único en lo que consiste la localización, en realidad también se incluyen cualquier tipo de adaptaciones culturales que haya que hacer al producto original para que se adapte correctamente a la cultura destino. Por ejemplo, es común en los diálogos de videojuegos que se utilicen juegos de palabras complejos, un buen localizador no ha de traducir literalmente estos juegos de palabras, sino que tendrá que buscar un equivalente adecuado a ellos en el nuevo idioma.


\section{Localization Quality Assurance}


\section{\textit{Errores} Linguísticos comunes} \label{bugs}

\subsection{Problema de fuente} \label{ErrorFuente}



	
\subsection{Implementación de texto incorrecta}\label{ErrorImpIncorrecta}

\subsection{Cadena no localizada}\label{ErrorNoLocalizada}

\subsection{Error tipográfico o de ortografía}\label{ErrorTypo}


\subsection{Error gramatical}\label{ErrorGramatical}


\subsection{Error de traducción}\label{ErrorTraducción}

\subsection{Solapamiento de texto}\label{ErrorSolapamiento}


\subsection{Truncamiento de texto}\label{ErrorTruncamiento}


\subsection{Error terminológico o incoherencia}\label{ErrorTermino}


\subsection{Incumplimiento de directrices}\label{ErrorDirectrices}


\subsection{Error de subtítulos}\label{ErrorSubtitulos}



\subsection{Error de audio}\label{ErrorAudio}



\subsection{Problemas culturales}\label{ErrorCultura}



\section{Reconocimiento Óptico de Caracteres (OCR)}
El OCR (Reconocimiento Óptico de Caracteres, por sus siglas en inglés Optical Character Recognition) es una tecnología que convierte imágenes de texto manuscrito, impreso o mecanografiado en datos de texto que las computadoras pueden interpretar y manipular. En otras palabras, puede extraer el texto contenida en una imagen.

Para llegar al objetivo de reconocer el texto y extraerlo de la imagen el OCR da una serie de pasos	\footnote{(Extracting text from an image using Ocropus) \url{https://www.danvk.org/2015/01/09/extracting-text-from-an-image-using-ocropus.html} }	\footnote{(Improving the quality of the output) \url{https://tesseract-ocr.github.io/tessdoc/ImproveQuality.html} }:
\begin{enumerate}
	\item \textbf{Preprocesamiento de la imagen:}
	
	El propósito de este paso es procesar la imagen para que sea más legible para la computadora y reconocer así de forma mas eficiente los textos.
	\begin{itemize}
		\item Conversión a escala de grises: La mayoría de los OCR primero convierten la imagen en blanco y negro o escala de grises para simplificar el procesamiento.
		\item Reducción de ruido: Se aplican filtros para eliminar manchas, borrones o marcas en la imagen que puedan afectar la precisión del reconocimiento.
		\item Binarización: La imagen se convierte a una representación en blanco y negro, donde los píxeles se clasifican como parte del fondo (blanco) o del texto (negro). Este proceso ayuda a aislar los caracteres.
	\end{itemize}
	
	\item \textbf{Segmentación:}
	
	El OCR divide la imagen en secciones manejables. Primero separa líneas de texto.  Luego separa las palabras y finalmente descompone las palabras en caracteres individuales. Este paso es crucial, ya que el OCR necesita reconocer los caracteres de manera individual, pero considerando también su contexto dentro de una palabra o frase.
	
	\item \textbf{Detección de características:}
	
	Extracción de características de los caracteres: El OCR analiza los caracteres y sus formas, midiendo varios atributos como las líneas, contornos, cruces de líneas, y la disposición de los píxeles. Estos datos son usados para diferenciar letras, números y símbolos similares (como ``O'' y ``0'' o ``l'' y ``1'').
	Se aplican técnicas basadas en modelos geométricos, estructuras de redes neuronales o de aprendizaje automático para identificar patrones comunes en los caracteres.
	
	\item \textbf{Reconocimiento del carácter:}
	
	Clasificación de los caracteres: Una vez identificadas las características, el OCR las compara con una base de datos o ``alfabeto'' interno de posibles caracteres. Esto puede hacerse de varias maneras, dependiendo del tipo de OCR\footnote{(¿Qué es el reconocimiento óptico de caracteres (OCR)?) \url{https://aws.amazon.com/es/what-is/ocr/} }:
	\begin{itemize}
		\item Métodos basados en plantillas: Se compara cada carácter con un conjunto de plantillas predefinidas. Si la forma del carácter coincide con una plantilla, se clasifica como ese carácter.
		\item Métodos basados en aprendizaje automático o redes neuronales: Las técnicas modernas de OCR suelen usar redes neuronales entrenadas con miles de ejemplos de texto. El sistema ``aprende'' a identificar caracteres y a hacer distinciones más sutiles en base a sus experiencias pasadas.
	\end{itemize}

	
	
\end{enumerate}
