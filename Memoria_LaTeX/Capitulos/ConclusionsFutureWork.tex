\chapter*{Conclusions and Future Work}
\label{cap:conclusions}
\addcontentsline{toc}{chapter}{Conclusions and Future Work}

Conclusions and future lines of work. This chapter contains the translation of Chapter \ref{cap:conclusiones}.

In conclusion, due to the significant evolution of the video game industry, there arises a need to adapt a video game to different language or region from the one in which it was originally developed. For this purpose, the processes of internationalization and localization are followed, where internationalization refers to providing support for different languages, and localization is the process in which the texts, graphics and game assets are translated and adapted.

To ensure that the game is ready to be published in other regions, \textit{Localization Quality Assurance} (LQA) is carried out to verify that there are no localization errors.

To assist the LQA process, a tool is designed where the user provides game screenshots and a configuration file, and the tool detects whether any localization errors exist. This tool consists of two parts: an OCR module and a test module.

For the OCR module, several OCR libraries are evaluated, and for our tool, Tesseract is used. To improve text recognition results, techniques such as image preprocessing are applied to better identify text in images, as well as methods like Levenshtein distance to remove noisy characters recognized by the OCR.

For the test module, tests are designed and implemented to detect localization errors. In our case, the implemented tests include text overlap, text truncation, and placeholder detection.

The tool will generate a result file that will later be interpreted to create a report.

The evaluation results of the tool were not very promising. There is a problem in recognizing text in complex images such as those from video games, which leads to poor test results due to the quality of the input data.

To improve the tool, the following future work is proposed:

\begin{enumerate}
	\item Image classification to detect the necessary preprocessing according to the characteristics of the images.
	\item Decision tree to configure a specific preprocessing pipeline based on image features.
	\item Integration of other OCR libraries as options for the user.
	\item Implementation of new tests to address other localization issues.
	\item Layout detection in images to recognize text regions and crop out irrelevant areas to improve recognition.
	\item A tool that captures video game screen data and generates the necessary assets for the localization error detection tool.
\end{enumerate}
