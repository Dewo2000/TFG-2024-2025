\chapter*{Conclusions and Future Work}
\label{cap:conclusions}
\addcontentsline{toc}{chapter}{Conclusions and Future Work}

Due to the significant growth of the video game industry, there is an increasing need to adapt games to languages and regions different from their original design. This is achieved through the processes of internationalization and localization, where internationalization refers to adding support for multiple languages, and localization is the process of translating and adapting texts, graphics, and other assets.

To ensure that a video game is ready to be released in other regions, the Localization Quality Assurance (LQA) process is applied. LQA is responsible for verifying that there are no localization-related errors.

To assist with the LQA process, a tool has been developed that allows users to provide screenshots from the game along with a configuration file. The tool will then detect whether any localization errors are present. This tool consists of two main modules: the OCR module and the testing module.

For the OCR module, several OCR libraries were evaluated, and Tesseract was selected for use in the tool. To improve text recognition accuracy, techniques such as image preprocessing are applied to enhance text clarity in images. Additionally, methods like the Levenshtein distance are used to filter out noisy characters incorrectly recognized by the OCR.

For the testing module, tests were designed and implemented to detect localization issues. In this project, the following tests were developed: text overlap detection, text truncation detection, and placeholder detection.

The tool generates a result file, which is later interpreted to produce a final report.

The evaluation results of the tool were acceptable, although there is clear room for improvement. The main limitation observed lies in the OCR’s ability to recognize text, especially in complex images such as those generated within video game environments.

To enhance the tool, the following future work is proposed:

\begin{enumerate}
	\item Image classification to determine the most appropriate preprocessing based on image characteristics.
	\item A decision tree to apply specific preprocessing steps depending on the input image’s features.
	\item Training the OCR model with the game’s specific font.
	\item Integration of alternative OCR libraries to provide users with more options.
	\item Implementation of additional tests to detect other types of localization issues.
	\item Layout detection in images to identify text areas and crop out irrelevant zones, improving recognition accuracy.
	\item A tool to automatically collect game screen data and generate the necessary assets for the localization error detection tool.
\end{enumerate}
