\chapter*{Resumen}

\section*{\tituloPortadaVal}

La localización es una tarea imprescindible de los videojuegos hoy en día, ya que los videojuegos se quieren vender y publicar en distintos países y culturas. Para ello, es necesario adaptar los videojuegos a diferentes lenguajes y a diferentes culturas. Ese proceso lo llamamos localización. El trabajo del equipo de localización puede llevar varias iteraciones  para evitar errores que puedan aparecer.

Hoy en día no existen programas que verifiquen si un videojuego tiene errores de localización, ya sean de traducción o de internacionalización, por lo que es necesario personal que tenga que hacer esa tarea de verificación, esto supone un alto coste en tiempo y dinero.

El objetivo de este trabajo es tratar de automatizar esas tareas empezando por reconocer y recoger el texto que aparece en el videojuego, seguido de unos tests que verifican si el texto reconocido tiene algún error de localización, generando así un documento que indique los posibles fallos de la localización.



\section*{Palabras clave}
   
\noindent Videojuegos, Localización, Internacionalización, Automatización, QA

   


