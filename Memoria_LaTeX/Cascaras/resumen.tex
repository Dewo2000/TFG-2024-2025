\chapter*{Resumen}

\section*{\tituloPortadaVal}

La localización es una tarea imprescindible de los videojuegos hoy en día ,ya que los videojuegos se quieren vender y publicar en distintos países y culturas,para ello es necesario adaptar nuestro videojuego a diferentes lenguajes y a diferentes cultura ,ese proceso lo llamamos localización.El trabajo del traductor puede llevar varias iteraciones por parte del traductor para evitar errores que pueda aparecer.

Hoy en día no existe programas que verifique si un videojuego esta bien traducido o tenga erratas tipográficas por lo que es necesario uno o varios personales tenga que hacer esa tarea de verificación, esto supone un trabajo costoso en tiempo y dinero.

Este trabajo se trata de automatizar esas tareas empezando por reconocer y recoger el texto que aparece en el videojuego , seguido de unos tests que verifica que si el texto reconocido tiene alguna errata o fallos de traducción generando así un documento que indique los posibles fallos de la localización.



\section*{Palabras clave}
   
\noindent Videojuegos,Localización,Internacionalización,Automatización,QA

   


