\chapter*{Abstract}

\section*{\tituloPortadaEngVal}

Localization is an essential task in modern video games, as these products are intended to be sold and published in various countries and cultures. To achieve this, video games must be adapted to different languages and cultural contexts. This process is known as localization. The work of the localization team often involves several iterations to avoid potential errors.

Currently, there are no tools that automatically verify whether a video game contains localization errors, whether related to translation or internationalization. As a result, this verification must be performed manually by specialized personnel, which involves high costs in terms of time and money.

The goal of this project is to automate these tasks, starting with the recognition and extraction of the text that appears in the video game. This is followed by a set of tests that check whether the recognized text contains any localization issues, ultimately generating a report that highlights potential localization errors.

\section*{Keywords}

\noindent Video games, Localization, Internationalization, Automation, QA, LQA, OCR, Tesseract.




